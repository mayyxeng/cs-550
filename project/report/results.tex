\section{Evaluation}
\label{sec:eval}

The authors implement the algorithm outlined in \secref{body} in a publicly available tool called \texttt{ct-verif}~\cite{ct-verif-github}. 
\texttt{ct-verif} takes as input the LLVM implementation of a cryptographic algorithm and outputs either proof of constant-timeliness or a counter-example showing how the property is violated. 
Under the covers, \texttt{ct-verif} leverages the SMACK verification tool~\cite{smack} to translate the LLVM to Boogie~\cite{boogie} code. 
It then performs its reduction on the Boogie code and applies the Boogie verifier to the resulting program. 

The authors evaluate their tool on examples from widely used cryptographic libraries including OpenSSL~\cite{openssl}, NaCl~\cite{nacl}, FourQlib~\cite{fourqlib} and \texttt{curve25519-donna}~\cite{donna}. The examples include encryption algorithms, hash functions, fixed-point arithmetic and elliptic-curve arithmetic. 
The size of the examples ranges from $50-1200$ lines of C code.

The authors demonstrate that \texttt{ct-verif} is typically able to verify that the examples are constant-time secure within seconds---most functions require $\leq 30s$ while the largest function requires $273s$. 
These results show that \texttt{ct-verif} can be easily integrated into the day-to-day development of cryptographic algorithms. 