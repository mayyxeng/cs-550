\section{Introduction}
% Say a few general words about the general context of the paper you chose. Explain why the topic is of interest, or where it can be applied. If it is about a piece of software or artifact, give a description of it. State the main result of the paper and why it is new or how it improves on previous state of knowledge. You can cite references using, for example \cite{BibliographyManagementLaTeX} and make a succint presentation of the organisation of your report.

Timing attacks are a family of side-channel attacks that extract information from software systems by measuring how long a program takes to execute under adversary-controlled inputs.
Constant-time (CT) programming techniques are used to harden programs against timing attacks.
CT programming involves rewriting a program such that (1)~its control flow does not depend on program secrets, (2)~its memory accesses do not depend on program secrets, and (3)~its variable-latency instructions do not depend on program secrets.

CT programming is the most effective software-based countermeasure against timing attacks.
However, writing CT programs is very difficult as it involves the use of low-level programming languages and programming practices that deviate from good software engineering principles.
CT code is therefore hard to read/reason about, and even expert-written ``CT'' programs have been shown to suffer from timing attacks.

Validating CT implementations is therefore a primary concern.
Extensive testing and code review cycles are insufficient to identify all CT implementation bugs as even production-grade software like Amazon's s2n library has managed to let multiple CT implemenation bugs go to a release.

Almeida et al. propose to use formal methods to assert the CT property of an input program.
This work's contributions include:
\begin{enumerate}[label=(\roman*)]
  \item a formal foundation to model constant-time programming properties,
  \item a sound and complete reduction from input programs to a safety assertion,
  \item an automated tool, \texttt{ct-verif}, that implements their reduction on LLVM code.
\end{enumerate}

Furthermore, CT programming in its purest form often suffers from a performance penalty.
test
